%\documentclass[12pt,twoside]{article}
\documentclass[conference]{IEEEtran}  % this should work with your LaTeX installation; else download extra package (www.ctan.org/pkg/ieeetran) or remove IEEE usage below

%%%%%%% Fill this out:
\newcommand{\trtitle}{Solving STRIPS Problems with Graphplan}
\newcommand{\titleshort}{Solving STRIPS with Graphplan} % title for header:
\newcommand{\authorlastnames}{Willmann} % alphabetical for seminars
\newcommand{\trcourse}{Knowledge Processing in Intelligent Systems: Practical Seminar}
\newcommand{\trgroup}{Knowledge Technology, WTM}
\newcommand{\truniversity}{Department of Informatics, University of Hamburg}
%%%%%%%%%%%%%%%%%%%%%%%%%%%%%%%%%%%%%%%%%%%%%%%%%%%%%%%%%%%%%
% Languages:

% If the thesis is written in English:
\usepackage[english]{babel} 						
\selectlanguage{english}

%%%%%%%%%%%%%%%%%%%%%%%%%%%%%%%%%%%%%%%%%%%%%%%%%%%%%%%%%%%%%
% Bind packages:
\usepackage{lipsum}

\usepackage{acronym}                    % Acronyms
\usepackage{algorithmic}								% Algorithms and Pseudocode
\usepackage{algorithm}									% Algorithms and Pseudocode
\usepackage{amsfonts}                   % AMS Math Packet (Fonts)
\usepackage{amsmath}                    % AMS Math Packet
\usepackage{amssymb}                    % Additional mathematical symbols
\usepackage{amsthm}
\usepackage{booktabs}                   % Nicer tables
%\usepackage[font=small,labelfont=bf]{caption} % Numbered captions for figures
\usepackage{color}                      % Enables defining of colors via \definecolor
\definecolor{uhhRed}{RGB}{254,0,0}		  % Official Uni Hamburg Red
\definecolor{uhhGrey}{RGB}{122,122,120} % Official Uni Hamburg Grey
\usepackage{fancybox}                   % Gleichungen einrahmen
\usepackage{fancyhdr}										% Packet for nicer headers
%\usepackage{fancyheadings}             % Nicer numbering of headlines

%\usepackage[outer=3.35cm]{geometry} 	  % Type area (size, margins...) !!!Release version
%\usepackage[outer=2.5cm]{geometry} 		% Type area (size, margins...) !!!Print version
%\usepackage{geometry} 									% Type area (size, margins...) !!!Proofread version
\usepackage{geometry} 	  % Type area (size, margins...) !!!Draft version
\geometry{a4paper,body={7.0in,9.1in}}

\usepackage{graphicx}                   % Inclusion of graphics
%\usepackage{latexsym}                  % Special symbols
\usepackage{longtable}									% Allow tables over several parges
\usepackage{listings}                   % Nicer source code listings
\usepackage{multicol}										% Content of a table over several columns
\usepackage{multirow}										% Content of a table over several rows
\usepackage{rotating}										% Alows to rotate text and objects
\usepackage[hang]{subfigure}            % Allows to use multiple (partial) figures in a fig
%\usepackage[font=footnotesize,labelfont=rm]{subfig}	% Pictures in a floating environment
\usepackage{tabularx}										% Tables with fixed width but variable rows
\usepackage{url,xspace,boxedminipage}   % Accurate display of URLs

%%%%%%%%%%%%%%%%%%%%%%%%%%%%%%%%%%%%%%%%%%%%%%%%%%%%%%%%%%%%%
% Configurationen:

\hyphenation{whe-ther} 									% Manually use: "\-" in a word: Staats\-ver\-trag

%\lstloadlanguages{C}                   % Set the default language for listings
\DeclareGraphicsExtensions{.pdf,.svg,.jpg,.png,.eps} % first try pdf, then eps, png and jpg
\graphicspath{{./src/}} 								% Path to a folder where all pictures are located
\pagestyle{fancy} 											% Use nicer header and footer

% Redefine the environments for floating objects:
\setcounter{topnumber}{3}
\setcounter{bottomnumber}{2}
\setcounter{totalnumber}{4}
\renewcommand{\topfraction}{0.9} 			  %Standard: 0.7
\renewcommand{\bottomfraction}{0.5}		  %Standard: 0.3
\renewcommand{\textfraction}{0.1}		  	%Standard: 0.2
\renewcommand{\floatpagefraction}{0.8} 	%Standard: 0.5

% Tables with a nicer padding:
\renewcommand{\arraystretch}{1.2}

%%%%%%%%%%%%%%%%%%%%%%%%%%%%
% Additional 'theorem' and 'definition' blocks:
\theoremstyle{plain}
\newtheorem{theorem}{Theorem}[section]
%\newtheorem{theorem}{Satz}[section]		% Wenn in Deutsch geschrieben wird.
\newtheorem{axiom}{Axiom}[section] 	
%\newtheorem{axiom}{Fakt}[chapter]			% Wenn in Deutsch geschrieben wird.
%Usage:%\begin{axiom}[optional description]%Main part%\end{fakt}

\theoremstyle{definition}
\newtheorem{definition}{Definition}[section]

%Additional types of axioms:
\newtheorem{lemma}[axiom]{Lemma}
\newtheorem{observation}[axiom]{Observation}

%Additional types of definitions:
\theoremstyle{remark}
%\newtheorem{remark}[definition]{Bemerkung} % Wenn in Deutsch geschrieben wird.
\newtheorem{remark}[definition]{Remark} 

%%%%%%%%%%%%%%%%%%%%%%%%%%%%
% Provides TODOs within the margin:
\newcommand{\TODO}[1]{\marginpar{\emph{\small{{\bf TODO: } #1}}}}

%%%%%%%%%%%%%%%%%%%%%%%%%%%%
% Abbreviations and mathematical symbols
\newcommand{\modd}{\text{ mod }}
\newcommand{\RS}{\mathbb{R}}
\newcommand{\NS}{\mathbb{N}}
\newcommand{\ZS}{\mathbb{Z}}
\newcommand{\dnormal}{\mathit{N}}
\newcommand{\duniform}{\mathit{U}}

\newcommand{\erdos}{Erd\H{o}s}
\newcommand{\renyi}{-R\'{e}nyi}
\usepackage{graphicx}

% correct bad hyphenation here
\hyphenation{}

%%%%%%%%%%%%%%%%%%%%%%%%%%%%%%%%%%%%%%%%%%%%%%%%%%%%%%%%%%%%%
% Document:
\begin{document}

%\title{\trtitle}
\title{\trtitle\\[1.5ex]
  \large \trcourse\\[0.5ex]
  \trgroup\\[0.5ex]
  \truniversity}


\renewcommand{\headheight}{14.5pt}

% Uncomment these lines if needed for header customization
%\fancyhead{}
%\fancyhead[LE]{  }
\fancyhead[LO]{\slshape \authorlastnames}
%\fancyhead[RE]{}
\fancyhead[RO]{ \slshape \titleshort}

% Author names and affiliations
\author{
  \IEEEauthorblockN{Sebastian Willmann}
  \IEEEauthorblockA{\textit{sebastian.willmann@studium-hamburg.de}}
}

% \IEEEauthorblockN{
%     John Adams,
%     Thomas Jefferson,
%     George Washington
% }

% \IEEEauthorblockA{
%     \{adams, jefferson, washington\}@informatik.uni-hamburg.de\\
% }

% \begin{tabular}{c}% Centering the institution information
% 	\trcourse\\
% 	\trgroup, \truniversity
% \end{tabular} 

% make the title area
\maketitle

\begin{abstract}
  Graphplan is an algorithm in the field of planning algorithms. 
  It works on a STRIPS (Stanford Research Institute Problem Solver) problem; 
  this problem format is introduced in section 1.
  % \lipsum[2]

\end{abstract}

\IEEEpeerreviewmaketitle

\section{STRIPS}
\label{sec:strips}

An introduction into the STRIPS problem format, with mathematical construction.

\subsection{Example}
\label{sec:strips_example}

A small, but nontrivial example of a STRIPS problem. 
The one on wikipedia (the monkey that moves the box to reach the bananas) is a bit too simple. 
It should take more than four steps to solve the problem. 

Optimally, it would also show the strengths of the Graphplan algorithm.

\subsection{Applications}
\label{sec:strips_applications}

A very short section about which problems the STRIPS problem format could be used for 
and why it can be especially useful (Reduction to P-Hard problems) % ? Graphplan claims to be P-Fast on a problem that is technically P-Space hard.

\section{Linear Search}
\label{sec:linear_search}

A small section on linear search, the search algorithm implemented
in the recommended Github repository. 
Should definitely be using terminology from the lecture.
(I'm pretty sure it's a breadth-first search, but I'm not sure; I haven't implemented it yet).

Also maybe a few notes on the implementation in LEAN. 

\subsection{Applying Linear Search}
\label{sec:linear_search_application}

A small section on applying linear search to the sample problem from
the section on STRIPS. Should get the reader to understand applied 
STRIPS a bit better and prepare for the Graphplan explaination.


\section{Graphplan}
\label{sec:graphplan}

A short introduction of Graphplan, with main idea and angle of attack.

\subsection{Functionalilty; Algorithm}
\label{sec:graphplan_functionality}

A longer section on how graphplan actually works, very mathematical. 
Probably even the mathematical description of the algorithm.
Again, with terminology from the lecture.


\subsection{Applying Graphplan}
\label{sec:graphplan_applying}

Applycation of graphplan to the same problem. 
Best if the visualization/problem shows the strengths of graphplan.

\section{Implementation in LEAN}
\label{sec:implementation}

A section (I don't yet know how long) about implementing STRIPS in Lean. 
Details, snags I've hit, performance, etc. 

Should mention the DSL for interactive solving. 

\subsection{Solving with LLM}
\label{sec:implementation_llm}

If wanted/necessary (since the topic list mentioned it) , a section on
getting an LLM to try to solve the problem, probably using the DSL for simplicity. 

\section{Conclusion}
\label{sec:conclusion}

Conclusion, if neccessary. 

% insert your bibliographic references into the bib.bib file
% \bibliographystyle{plain}
% \addcontentsline{toc}{section}{Bibliography}% Add to the TOC
% \bibliography{bib}
\end{document}
